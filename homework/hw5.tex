\documentclass{article}
\usepackage{fullpage,graphicx}
\usepackage{amsmath,amsfonts,amsthm,amssymb,multirow,xcolor}
\usepackage{algorithmic}
\usepackage[ruled,vlined,commentsnumbered,titlenotnumbered]{algorithm2e}
\usepackage{minted}
\usepackage{multicol}
\usepackage{tikz}
\usetikzlibrary{positioning}


\usepackage{pgfplots}
\pgfplotsset{compat=1.10}
\usetikzlibrary{shapes.geometric,arrows,fit,matrix,positioning}
\tikzset
{
  treenode/.style = {circle, draw=black, align=center, minimum size=1cm},
  subtree/.style  = {isosceles triangle, draw=black, align=center, minimum height=0.5cm, minimum width=1cm, shape border rotate=90, anchor=north}
}

\begin{document}
\noindent
CS 161 \hfill \textbf{Homework 5} \newline 
Summer 2018 \hfill \textbf{Due:} Friday 8/10 at 5 p.m. on Gradescope

\noindent\rule{\linewidth}{0.4pt}

\section*{Exercises}

Exercises should be completed \textbf{on your own}.

\noindent\rule{\linewidth}{1.0pt}

\begin{enumerate}
  \item \textbf{(Warmup with MSTs)} \textbf{(7 pt.)}

    In all of the following questions, assume that $G$ is an undirected,
    connected graph on $n$ nodes with distinct, positive edge weights. Also
    assume that $G$ is simple (no self-loops and no multi-edges) in all parts.

    You might find the following ``cycle property'' useful and can reference it
    without proof (although you might want to convince yourself of it!): The
    most expensive edge on any cycle does not belong to MST (when edge weights
    are distinct).
    
    \begin{enumerate}
      \item \textbf{(1 pt.)} Consider the edges of $G$ sorted in increasing
        order of weight and place the first $n-1$ edges into a set $S$. In
        other words, $S$ is the set of $n-1$ edges with the smallest weights.
    
        Prove or disprove: $S$ is a minimum spanning tree in $G$ (assume
        $n \geq 2$).
    
        \textbf{[We are expecting: Either a formal proof or a counterexample.]}
    \item \textbf{(2 pt.)} Prove or disprove: There is a unique minimum
      spanning tree in $G$.
    
        \textbf{[We are expecting: Either a formal proof or a counterexample.]}
    \item \textbf{(4 pt.)} Let $T$ be a minimum spanning tree of $G$ and $e$
      be some edge in $G$ (which may or may not belong to $T$). Suppose we
      obtain a new graph $G'$ from $G$ by preserving everything the same except
      that we decrease the weight of $e$ (but assume that the weights of all
      edges in $G'$ are still distinct). Design an algorithm that can find a
      minimum spanning tree $T'$ of $G'$ in $O(|V|)$-time, given $G$, $T$, $e$
      and its newly decreased weight $w$. Prove the correctness and analyze the
      runtime.
    
      \textbf{[We are expecting: An English description of the algorithm, and
      brief justification of its runtime.]}
    \end{enumerate}
\end{enumerate}

\newpage

\section*{Problems}

You can collaborate with your classmates about the problems. However:

\begin{itemize}
  \item Try the problems on your own \textit{before} collaborating.
  \item Write up your solutions yourself, in your own words.  You should never
    share your typed-up solutions with your collaborators.
  \item If you collaborated, list the names of the students you collaborated
    with at the beginning of each problem.
\end{itemize}

\noindent\rule{\linewidth}{1.0pt}

\begin{enumerate}
  \item \textbf{(k-friendliness)} \textbf{(4 pt.)}

    Suppose you model the friendships of students in CS 161 as an undirected
    graph $G = (V,E)$ (i.e. there is an edge between students A and B if they
    are friends). For any integer $k$, we say that a group of students,
    $S \subset V$, is ``$k$-friendly'' if every student in $S$ has at least $k$
    friends in $S$. Design an $O(|V|+|E|)$-time algorithm which, given the graph
    $G$ and an integer $k > 0$, finds the largest set $S \subset V$ that is
    $k$-friendly; if no such set exists, then the algorithm should say so.
  
    [\textbf{We are expecting: An English description of the algorithm and
    brief justification of its runtime.}]
  
  \item \textbf{(Placing receivers)} \textbf{(11 pt.)}

    Suppose there are $n$ transmitters fixed in place along a linear track.
    The $i$'th transmitter has communication range $[a_i, b_i]$, for
    $a_i \leq b_i$: that is, any receiver placed within the range $[a_i, b_i]$
    can receive signals from the $i$'th transmitter. Assume that the
    transmitters are sorted by the right endpoint of their communication range:
    that is, if $i < j$, then $b_i \leq b_j$.

    We want to pick a set of points on the track to place receivers such that
    we can receive signals from every transmitter while minimizing the number
    of receivers necessary. In other words, we want to find a minimum set of
    points $S$ on the line such that for every transmitter $i$, there is some
    receiver $s \in S$ such that $a_i \leq s \leq b_i$. 

    In this problem, we describe a \textbf{greedy algorithm} that finds the
    minimum set $S$ of receiver locations in expected time $O(n)$, and prove
    that it is correct.

    Suppose the greedy algorithm which works as follows: we place receivers
    one at a time. At each step, suppose that $i^*$ is the smallest $i$ so that
    transmitter $i$ cannot be heard by any receiver placed so far, and place a
    receiver at $b_{i^*}$. Continue placing receivers in this way until all the
    transmitters can be heard.

    \begin{enumerate}
      \item \textbf{(2 pt.)} Based on this English description, write
        pseudocode to implement the algorithm in time $O(n)$. Assume the input
        to your algorithm is two arrays, $a$, and $b$, which contain the values
        of $a_i$ and $b_i$ in the order described above.

        \textbf{[We are expecting detailed pseudocode, and an informal
        justification of the running time.]}

      \item \textbf{(9 pt.)} Prove that this algorithm is correct (in terms of
        both legality and optimality), following the outline below.

        We recommend induction on $t$, using a greedy stays ahead or greedy
        exchange argument.

        \begin{enumerate}
          \item \textbf{(3 pt.)} State your inductive hypothesis.
          \item \textbf{(1 pt.)} Prove the base case.
          \item \textbf{(4 pt.)} Prove the inductive step. 
          \item \textbf{(1 pt.)} Finish the argument: once the induction
            argument is complete, show that this implies that the algorithm is
            correct.
        \end{enumerate}

        \textbf{[We are expecting: For (i), a statement of an inductive
          hypothesis. For (ii), (iii), (iv), we are expecting a formal proof,
          including a statement of what it is you are proving.]}
    \end{enumerate}
  \item \textbf{(Min Gradespan)} \textbf{(4 pt.)}

    An analog of the following problem arises when you are trying to assign
    jobs to servers; while the greedy algorithm will not be optimal, it comes
    pretty close.

    Suppose the course staff has $n$ exams to grade, and there are $k$ course
    assistants (CAs) who will be doing the grading. We can all tell the exact
    amount of time it will take to grade a given exam (from glancing at the
    handwriting and text density), and all the CAs would take this exact amount
    of time to grade that exam. I want to divide the set of $n$ exams among the
    CAs so as to minimize the maximum amount of time it will take any of the
    CAs to grade (i.e. we will all wait until the last CA is done grading their
    pile, and I want to minimize the total time from when we all start grading,
    until this last CA is done.)

    Suppose I start with the stack of exams, and want to divide the exams into
    the $k$ stacks by stepping through the pile of exams just once---namely,
    when I look at the $i^{\text{th}}$ exam, I will immediately know how long
    it takes to grade, and then assign it to one of the $k$ CA piles. Suppose I
    do this by keeping track of the total grading time of each stack, and
    assigning the next exam to the stack that currently has the shortest
    grading time.
    \begin{enumerate}
      \item \textbf{(1 pt.)} Describe a concrete instance of this problem
        involving $n=3$ exams and $k=2$ CAs that illustrates that this
        algorithm will not always result in the optimal allocation.

        Specifically, give a list of the three grading times such that the
        greedy algorithm will result in one of the two CAs spending longer than
        necessary, and describe both the optimal allocation of the exams to the
        two piles, and the allocation that the greedy algorithm will choose.

        \textbf{[We are expecting: An example.]}
      \item \textbf{(3 pt.)} Prove that the allocation given by the greedy
        algorithm will be suboptimal by \emph{at most} a factor of 2. Namely,
        if there exists an allocation of the $n$ exams into $k$ piles such that
        all CAs can finish grading within time $t_{opt},$ prove that in the
        allocation chosen by the greedy algorithm, the maximum time will be at
        most $2\cdot t_{opt}.$ [Hint: what can you say about the amount of
        grading time that the longest-working CA would have had, if the last
        exam did not exist?]

        \textbf{[We are expecting: A formal proof.]}
      \item \textbf{(0 pt., Food for thought)} How much can you improve this
        factor-of-two ratio of greedy and opt (i.e. this factor of 2
        ``competitive ratio'' of the greedy algorithm) if you sort the exams
        in decreasing order of grading time before running the greedy
        algorithm?

        \textbf{[We are not expecting anything.]}
\end{enumerate}
\end{enumerate}

\end{document}